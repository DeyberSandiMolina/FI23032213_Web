\documentclass[12pt,a4paper]{article}
\usepackage[spanish]{babel}
\usepackage[utf8]{inputenc}
\usepackage{graphicx}
\usepackage{hyperref}
\usepackage{pgf}
\usepackage{tikz}
\usetikzlibrary{positioning, shapes, arrows}
\usepackage[margin=2.5cm]{geometry}

\title{Quixo Web - Proyecto Final}
\author{Deyber Sandi Molina \\ FI23032213}
\date{\today}

\begin{document}

\maketitle

\begin{center}
\textbf{Proyecto individual desarrollado para el curso de Programación Web}
\end{center}

\section*{Información del Estudiante}

\begin{table}[h]
\centering
\begin{tabular}{|l|l|}
\hline
\textbf{Campo} & \textbf{Información} \\
\hline
Nombre completo & Deyber Sandi Molina \\
Carné & FI23032213 \\
Correo institucional & dsandi30398@ufide.ac.cr \\
Usuario Git & DeyberSandiMolina \\
\hline
\end{tabular}
\end{table}

\section*{Información del Proyecto}

\subsection*{Frameworks y Herramientas Utilizadas}
\begin{itemize}
    \item \textbf{Backend:} ASP.NET Core 8.0
    \item \textbf{Frontend:} HTML5, CSS3, JavaScript (Vanilla)
    \item \textbf{Base de Datos:} SQLite (Entity Framework Core)
    \item \textbf{ORM:} Entity Framework Core 8.0
    \item \textbf{IDE:} Visual Studio 2022 / Visual Studio Code
    \item \textbf{Gestión de Paquetes:} NuGet
    \item \textbf{Control de Versiones:} GitHub
    \item \textbf{Sistema Operativo:} Windows 11
\end{itemize}

\subsection*{Tipo de Aplicación}
\textbf{MPA (Multi-Page Application)} - Aplicación web tradicional con múltiples páginas HTML implementada con ASP.NET Core MVC.

\subsection*{Arquitectura}
\textbf{MVC (Model-View-Controller)} con separación de responsabilidades:

\begin{figure}[h]
\centering
\begin{tikzpicture}[
    node distance=1cm,
    block/.style={rectangle, draw=black, thick, minimum width=3cm, minimum height=1cm, text centered, rounded corners},
    arrow/.style={->, >=stealth, thick}
]

\node[block, fill=blue!20] (controller) {Controladores};
\node[block, fill=green!20, below=of controller] (services) {Servicios};
\node[block, fill=red!20, below left=of services] (models) {Modelos};
\node[block, fill=yellow!20, below right=of services] (views) {Vistas};
\node[block, fill=purple!20, below=of services] (repository) {Repositorios};
\node[block, fill=orange!20, below=of repository] (database) {Base de Datos};

\draw[arrow] (controller) -- node[right] {Solicitudes} (services);
\draw[arrow] (services) -- node[left] {Datos} (models);
\draw[arrow] (services) -- node[right] {DTOs} (views);
\draw[arrow] (services) -- node[right] {Operaciones CRUD} (repository);
\draw[arrow] (repository) -- node[right] {Consultas SQL} (database);
\draw[arrow] (views) -- node[above] {Respuesta HTML} (controller);
\draw[arrow] (models) -- node[below] {Mapeo} (repository);

\end{tikzpicture}
\caption{Arquitectura MVC del proyecto Quixo Web}
\end{figure}

\begin{figure}[h]
\centering
\resizebox{\textwidth}{!}{%
\begin{tikzpicture}[
    node distance=1.5cm and 1.5cm,
    table/.style={
        rectangle,
        draw=black,
        thick,
        minimum width=4cm,
        minimum height=2cm,
        text centered,
        fill=blue!10,
        rounded corners=5pt
    },
    pk/.style={fill=yellow!30},
    fk/.style={fill=green!30}
]

% ========== GAMES ENTITY ==========
\node[table, pk] (games) at (0,0) {
    \textbf{GAMES} \\
    \scriptsize GameId (PK) \\
    \scriptsize Mode (2=2J, 4=4J) \\
    \scriptsize CreatedAt \\
    \scriptsize EndedAt \\
    \scriptsize TotalTime \\
    \scriptsize WinnerPlayerId (FK) \\
    \scriptsize WinnerTeamId (FK)
};

% ========== PLAYERS ENTITY ==========
\node[table, pk] (players) at (-5,-4) {
    \textbf{PLAYERS} \\
    \scriptsize PlayerId (PK) \\
    \scriptsize Name \\
    \scriptsize PlayerOrder \\
    \scriptsize GameId (FK) \\
    \scriptsize TeamId (FK)
};

% ========== TEAMS ENTITY ==========
\node[table, pk] (teams) at (5,-4) {
    \textbf{TEAMS} \\
    \scriptsize TeamId (PK) \\
    \scriptsize Name \\
    \scriptsize GameId (FK) \\
    \scriptsize Player1Id (FK) \\
    \scriptsize Player2Id (FK) \\
    \scriptsize GamesWon
};

% ========== MOVES ENTITY ==========
\node[table, pk] (moves) at (-2.5,-8) {
    \textbf{MOVES} \\
    \scriptsize MoveId (PK) \\
    \scriptsize GameId (FK) \\
    \scriptsize PlayerId (FK) \\
    \scriptsize TurnNumber \\
    \scriptsize CubeTakenRow (0-4) \\
    \scriptsize CubeTakenCol (0-4) \\
    \scriptsize CubePlacedRow (0-4) \\
    \scriptsize CubePlacedCol (0-4) \\
    \scriptsize Symbol (X/O) \\
    \scriptsize DotDirection (0-3)
};

% ========== BOARDSTATES ENTITY ==========
\node[table, pk] (boardstates) at (2.5,-8) {
    \textbf{BOARDSTATES} \\
    \scriptsize BoardStateId (PK) \\
    \scriptsize GameId (FK) \\
    \scriptsize MoveId (FK) \\
    \scriptsize TurnNumber \\
    \scriptsize StateJson \\
    \scriptsize CreatedAt
};

% ========== RELACIONES ==========
% Game → Players (1:N)
\draw[->, thick, blue] (games.south west) -- node[left, font=\scriptsize, pos=0.7] {1} (players.north);
\draw[<-, thick, blue] (players.north) -- node[right, font=\scriptsize, pos=0.3] {N} (games.south west);

% Game → Teams (1:N)
\draw[->, thick, red] (games.south east) -- node[right, font=\scriptsize, pos=0.7] {1} (teams.north);
\draw[<-, thick, red] (teams.north) -- node[left, font=\scriptsize, pos=0.3] {N} (games.south east);

% Team → Players (1:2)
\draw[<->, thick, green!60!black, dashed] (teams.south) -- node[left, font=\scriptsize] {1} (players.east);
\draw[<->, thick, green!60!black, dashed] (players.east) -- node[right, font=\scriptsize] {2} (teams.south);

% Game → Moves (1:N)
\draw[->, thick, purple] (games.south) -- node[left, font=\scriptsize, pos=0.7] {1} (moves.north);
\draw[<-, thick, purple] (moves.north) -- node[right, font=\scriptsize, pos=0.3] {N} (games.south);

% Player → Moves (1:N)
\draw[->, thick, orange] (players.south) -- node[left, font=\scriptsize, pos=0.7] {1} (moves.west);
\draw[<-, thick, orange] (moves.west) -- node[right, font=\scriptsize, pos=0.3] {N} (players.south);

% Game → BoardStates (1:N)
\draw[->, thick, brown] (games.south) -- node[right, font=\scriptsize, pos=0.7] {1} (boardstates.north);
\draw[<-, thick, brown] (boardstates.north) -- node[left, font=\scriptsize, pos=0.3] {N} (games.south);

% Move → BoardState (0:1)
\draw[<->, thick, gray, dashed] (moves.east) -- node[above, font=\scriptsize] {0..1} (boardstates.west);

% Game → WinnerPlayer (1:1)
\draw[->, thick, cyan, dashed] (games) -- ++(0,1.5) -| node[above, font=\scriptsize] {ganador} (players);

% Game → WinnerTeam (1:1)
\draw[->, thick, magenta, dashed] (games) -- ++(0,1.5) -| node[above, font=\scriptsize] {ganador} (teams);

\end{tikzpicture}
}
\end{figure}

\subsection*{Especificaciones de Tablas}

\subsubsection*{Games (Juegos)}
\begin{tabular}{|l|l|l|p{7.5cm}|}
\hline
\textbf{Columna} & \textbf{Tipo} & \textbf{Restricción} & \textbf{Descripción} \\
\hline
GameId & INT & PRIMARY KEY & Identificador único del juego \\
Mode & INT & NOT NULL, CHECK (2,4) & Modo: 2=2 jugadores, 4=4 jugadores \\
CreatedAt & DATETIME & NOT NULL & Fecha y hora de creación \\
EndedAt & DATETIME & NULLABLE & Fecha y hora de finalización \\
TotalTime & TIME & NOT NULL & Duración total de la partida \\
WinnerPlayerId & INT & NULLABLE, FK & ID del jugador ganador (modo 2J) \\
WinnerTeamId & INT & NULLABLE, FK & ID del equipo ganador (modo 4J) \\
\hline
\end{tabular}

\subsubsection*{Players (Jugadores)}
\begin{tabular}{|l|l|l|p{7.5cm}|}
\hline
\textbf{Columna} & \textbf{Tipo} & \textbf{Restricción} & \textbf{Descripción} \\
\hline
PlayerId & INT & PRIMARY KEY & Identificador único del jugador \\
Name & VARCHAR(50) & NOT NULL & Nombre del jugador \\
PlayerOrder & INT & NOT NULL, CHECK (1-4) & Orden de turno: 1,2,3,4 \\
GameId & INT & FOREIGN KEY NOT NULL & Juego al que pertenece \\
TeamId & INT & FOREIGN KEY NULLABLE & Equipo asignado (solo modo 4J) \\
\hline
\end{tabular}

\subsubsection*{Teams (Equipos)}
\begin{tabular}{|l|l|l|p{7.5cm}|}
\hline
\textbf{Columna} & \textbf{Tipo} & \textbf{Restricción} & \textbf{Descripción} \\
\hline
TeamId & INT & PRIMARY KEY & Identificador único del equipo \\
Name & VARCHAR(50) & NOT NULL & Nombre del equipo (ej: "Equipo A") \\
GameId & INT & FOREIGN KEY NOT NULL & Juego al que pertenece \\
Player1Id & INT & FOREIGN KEY NOT NULL & Primer jugador del equipo \\
Player2Id & INT & FOREIGN KEY NOT NULL & Segundo jugador del equipo \\
GamesWon & INT & DEFAULT 0 & Número de partidas ganadas \\
\hline
\end{tabular}

\subsubsection*{Moves (Movimientos)}
\begin{tabular}{|l|l|l|p{7.5cm}|}
\hline
\textbf{Columna} & \textbf{Tipo} & \textbf{Restricción} & \textbf{Descripción} \\
\hline
MoveId & INT & PRIMARY KEY & Identificador único del movimiento \\
GameId & INT & FOREIGN KEY NOT NULL & Juego al que pertenece \\
PlayerId & INT & FOREIGN KEY NULLABLE & Jugador que realizó el movimiento \\
TurnNumber & INT & NOT NULL, CHECK (>0) & Número de turno en la partida \\
CubeTakenRow & INT & NOT NULL, CHECK (0-4) & Fila del cubo tomado \\
CubeTakenCol & INT & NOT NULL, CHECK (0-4) & Columna del cubo tomado \\
CubePlacedRow & INT & NOT NULL, CHECK (0-4) & Fila donde se coloca el cubo \\
CubePlacedCol & INT & NOT NULL, CHECK (0-4) & Columna donde se coloca el cubo \\
Symbol & CHAR(1) & NOT NULL, CHECK (X,O) & Símbolo: 'X' o 'O' \\
DotDirection & INT & NULLABLE, CHECK (0-3) & Orientación del punto (0-3) modo 4J \\
\hline
\end{tabular}

\subsubsection*{BoardStates (Estados del Tablero)}
\begin{tabular}{|l|l|l|p{7.5cm}|}
\hline
\textbf{Columna} & \textbf{Tipo} & \textbf{Restricción} & \textbf{Descripción} \\
\hline
BoardStateId & INT & PRIMARY KEY & Identificador único del estado \\
GameId & INT & FOREIGN KEY NOT NULL & Referencia al juego \\
MoveId & INT & FOREIGN KEY NULLABLE & Movimiento asociado (opcional) \\
TurnNumber & INT & NOT NULL & Número de turno correspondiente \\
StateJson & TEXT & NOT NULL & Estado completo del tablero en JSON \\
CreatedAt & DATETIME & NOT NULL & Fecha y hora del estado \\
\hline
\end{tabular}

\subsection*{Relaciones y Cardinalidades}

\begin{itemize}
    \item \textbf{Games → Players (1:N):} Un juego tiene múltiples jugadores (2 o 4)
    \item \textbf{Games → Teams (1:N):} Un juego tiene múltiples equipos (0 o 2)
    \item \textbf{Teams → Players (1:2):} Un equipo tiene exactamente 2 jugadores
    \item \textbf{Games → Moves (1:N):} Un juego tiene múltiples movimientos
    \item \textbf{Players → Moves (1:N):} Un jugador realiza múltiples movimientos
    \item \textbf{Games → BoardStates (1:N):} Un juego tiene múltiples estados de tablero
    \item \textbf{Moves → BoardStates (0:1):} Un movimiento puede tener un estado asociado
    \item \textbf{Games → WinnerPlayer (1:1):} Relación opcional para jugador ganador
    \item \textbf{Games → WinnerTeam (1:1):} Relación opcional para equipo ganador
\end{itemize}

\subsection*{Restricciones de Integridad}
\begin{itemize}
    \item \textbf{Modo consistente:} Si Mode=2, Teams debe estar vacío
    \item \textbf{Modo consistente:} Si Mode=4, cada Team debe tener Player1Id y Player2Id
    \item \textbf{Unicidad:} PlayerOrder único por GameId
    \item \textbf{Validación:} Coordenadas entre 0-4 para el tablero 5x5
    \item \textbf{Validación:} Símbolos solo 'X' o 'O'
\end{itemize}

\section*{Referencias y Recursos Utilizados}

\subsection*{Documentación Oficial}
\begin{itemize}
    \item \textbf{ASP.NET Core Documentation:} \url{https://docs.microsoft.com/aspnet/core}
    \item \textbf{Entity Framework Core Documentation:} \url{https://docs.microsoft.com/ef/core}
    \item \textbf{SQLite Documentation:} \url{https://www.sqlite.org/docs.html}
    \item \textbf{Bootstrap 5 Documentation:} \url{https://getbootstrap.com/docs/5.0}
\end{itemize}

\subsection*{Tutoriales y Guías}
\begin{itemize}
    \item \textbf{Build a Web App with ASP.NET Core MVC:} \url{https://learn.microsoft.com/training/paths/aspnet-core-web-app}
    \item \textbf{Getting Started with EF Core:} \url{https://learn.microsoft.com/ef/core/get-started/overview/first-app}
    \item \textbf{JavaScript DOM Manipulation:} \url{https://developer.mozilla.org/en-US/docs/Web/API/Document_Object_Model}
    \item \textbf{SQLite with .NET:} \url{https://learn.microsoft.com/dotnet/standard/data/sqlite}
\end{itemize}

\subsection*{Stack Overflow - Consultas Resueltas}
\begin{itemize}
    \item \textbf{Relaciones en Entity Framework Core:} Problemas con claves foráneas y relaciones uno-a-muchos
    \item \textbf{Manejo de sesiones en ASP.NET Core:} Cómo mantener estado entre peticiones
    \item \textbf{JavaScript para juegos de tablero:} Implementación de lógica de juego en cliente
    \item \textbf{Exportación a XML en C\#:} Generación de archivos XML desde objetos
\end{itemize}

\section*{Uso de Asistentes de IA}

\subsection*{DeepSeek - Asistencia en Desarrollo}
Durante el desarrollo del proyecto, se utilizó el asistente de IA DeepSeek para resolver problemas específicos y optimizar código:

\begin{verbatim}
Prompts principales utilizados:

1. "Ayuda con el diseño de base de datos para un juego que soporte 
   tanto modo 2 jugadores como modo 4 jugadores con equipos"

2. "Problema con Entity Framework Core: las relaciones no se 
   persisten correctamente en SQLite"

3. "Cómo crear un reloj de 7 segmentos en JavaScript para mostrar 
   el tiempo transcurrido en la partida"

4. "Exportar datos de juego a formato XML en ASP.NET Core MVC"

\end{verbatim}

\subsection*{Código Generado con Asistencia de IA}
\begin{itemize}
    \item \textbf{QuixoRepository.cs:} Patrón Repository para acceso a datos
    \item \textbf{HistoryController.cs:} Controlador para historial y exportación
    \item \textbf{StatsController.cs:} Controlador para estadísticas
    \item \textbf{JavaScript del reloj 7 segmentos:} Visualización de tiempo
\end{itemize}

\section*{Instrucciones de Instalación y Ejecución}

\subsection*{Requisitos del Sistema}
\begin{itemize}
    \item \textbf{Sistema Operativo:} Windows 10/11, macOS, o Linux
    \item \textbf{.NET SDK:} Versión 8.0 o superior
    \item \textbf{Editor de Código:} Visual Studio 2022, VS Code, o similar
    \item \textbf{Navegador Web:} Chrome, Firefox, Edge (versiones recientes)
    \item \textbf{Memoria RAM:} Mínimo 4GB (recomendado 8GB)
    \item \textbf{Espacio en disco:} 500MB libres
\end{itemize}

\subsection*{Instalación Paso a Paso}

\begin{enumerate}
    \item \textbf{Clonar el repositorio:}
    \begin{verbatim}
    git clone https://github.com/DeyberSandiMolina/FI23032213_Web.git
    cd Proyecto/QuixoWeb
    \end{verbatim}
    
    \item \textbf{Verificar instalación de .NET:}
    \begin{verbatim}
    dotnet --version
    # Debe mostrar 8.0.x o superior
    \end{verbatim}
    
    \item \textbf{Restaurar dependencias:}
    \begin{verbatim}
    dotnet restore
    \end{verbatim}
    
    \item \textbf{Configurar base de datos:}
    \begin{verbatim}
    dotnet ef database update
    # O simplemente ejecutar la aplicación (se crea automáticamente)
    \end{verbatim}
\end{enumerate}

\subsection*{Compilación}
\begin{verbatim}
dotnet build
\end{verbatim}

\subsection*{Ejecución}
\begin{verbatim}
dotnet run
\end{verbatim}

\noindent La aplicación estará disponible en: \url{http://localhost:5267}

\subsection*{Estructura del Proyecto}
\begin{verbatim}
QuixoWeb/
├── Controllers/           # Controladores ASP.NET Core
│   ├── GameController.cs      # Control del juego
│   ├── HistoryController.cs   # Historial y exportación
│   ├── StatsController.cs     # Estadísticas
│   └── HomeController.cs      # Página principal
├── Models/                # Entidades de EF Core
│   ├── Game.cs               # Entidad Juego
│   ├── Move.cs               # Entidad Movimiento
│   ├── Player.cs             # Entidad Jugador
│   ├── Team.cs               # Entidad Equipo
│   └── BoardState.cs         # Estado del tablero
├── Views/                 # Vistas Razor
│   ├── Game/
│   │   ├── State.cshtml      # Vista del juego
│   │   └── Create.cshtml     # Crear juego
│   ├── History/
│   │   ├── Index.cshtml      # Lista de partidas
│   │   └── ViewGame.cshtml   # Ver partida específica
│   ├── Stats/
│   │   └── Index.cshtml      # Estadísticas
│   └── Home/
│       └── Index.cshtml      # Página principal
├── Services/              # Lógica de negocio
│   └── GameService.cs        # Servicio principal
├── Data/                  # Acceso a datos
│   ├── Repositories/
│   │   └── QuixoRepository.cs # Patrón Repository
│   └── QuixoDbContext.cs     # Contexto de BD
├── Domain/                # Dominio del juego
│   └── GameEngine.cs         # Motor del juego
├── Application/           # Capa de aplicación
│   └── DTOs/                 # Data Transfer Objects
├── wwwroot/              # Archivos estáticos
│   ├── css/                  # Estilos CSS
│   └── js/                   # JavaScript
├── Properties/           # Configuraciones
├── appsettings.json     # Configuración aplicación
├── Program.cs           # Punto de entrada
├── QuixoWeb.csproj      # Archivo proyecto
├── README.tex           # Este documento
├── README.pdf           # PDF generado
└── README.md            # README para GitHub
\end{verbatim}

\section*{Pruebas y Verificación}

\subsection*{Pruebas de Funcionalidad}
\begin{enumerate}
    \item \textbf{Creación de juego:}
    \begin{itemize}
        \item Acceder a \url{http://localhost:5267}
        \item Click en "Jugar una partida nueva"
        \item Seleccionar modo 2 o 4 jugadores
        \item Verificar que se crea correctamente
    \end{itemize}
    
    \item \textbf{Juego básico:}
    \begin{itemize}
        \item Realizar movimientos válidos
        \item Verificar que se alternan turnos
        \item Comprobar que el reloj funciona
        \item Intentar movimientos inválidos (deben rechazarse)
    \end{itemize}
    
    \item \textbf{Detección de ganador:}
    \begin{itemize}
        \item Completar una línea de 5 símbolos
        \item Verificar que se muestra mensaje de victoria
        \item Comprobar que se deshabilita el tablero
    \end{itemize}
    
    \item \textbf{Historial:}
    \begin{itemize}
        \item Ir a "Partidas finalizadas"
        \item Ver lista de juegos completados
        \item Click en "Ver" para una partida
        \item Navegar con botones anterior/siguiente
        \item Verificar que muestra cada movimiento
    \end{itemize}
    
    \item \textbf{Exportación:}
    \begin{itemize}
        \item En historial, click en "Exportar XML"
        \item Verificar que se descarga archivo .xml
        \item Abrir archivo y comprobar formato
    \end{itemize}
    
    \item \textbf{Estadísticas:}
    \begin{itemize}
        \item Ir a "Estadísticas"
        \item Ver tablas de efectividad
        \item Comprobar que los cálculos son correctos
    \end{itemize}
\end{enumerate}

\subsection*{Casos de Prueba Específicos}
\begin{itemize}
    \item \textbf{Caso 1:} Modo 2 jugadores, victoria normal
    \item \textbf{Caso 2:} Modo 2 jugadores, derrota accidental
    \item \textbf{Caso 3:} Modo 4 jugadores, juego completo
    \item \textbf{Caso 4:} Exportación de partida con muchos movimientos
    \item \textbf{Caso 5:} Estadísticas después de múltiples partidas
\end{itemize}

\section*{Características Implementadas}

\subsection*{Completadas ✅}
\begin{itemize}
    \item \textbf{Juego básico:} Tablero 5x5, movimientos válidos
    \item \textbf{Dos modos:} 2 jugadores y 4 jugadores (equipos)
    \item \textbf{Detección de ganador:} Horizontal, vertical, diagonal
    \item \textbf{Historial completo:} Lista de todas las partidas
    \item \textbf{Navegación por movimientos:} Ver partida turno por turno
    \item \textbf{Reloj 7 segmentos:} Tiempo real y en historial
    \item \textbf{Estadísticas:} Efectividad de jugadores/equipos
    \item \textbf{Exportación XML:} Guardar partida completa
    \item \textbf{Base de datos:} Persistencia SQLite
    \item \textbf{Interfaz responsiva:} Funciona en móviles y desktop
\end{itemize}

\subsection*{Características Avanzadas}
\begin{itemize}
    \item \textbf{Validación en tiempo real:} Movimientos inválidos bloqueados
    \item \textbf{Estado en memoria:} Juegos activos cargados dinámicamente
    \item \textbf{Reconstrucción de partidas:} Recargar cualquier partida desde BD
    \item \textbf{Cálculo automático:} Tiempos, estadísticas, efectividad
    \item \textbf{Mensajes descriptivos:} Feedback claro al usuario
\end{itemize}

\section*{Aprendizajes y Conclusiones}

\subsection*{Aprendizajes Técnicos}
\begin{itemize}
    \item \textbf{ASP.NET Core MVC:} Arquitectura completa de aplicaciones web
    \item \textbf{Entity Framework Core:} ORM y manejo de bases de datos
    \item \textbf{SQLite:} Base de datos embebida para aplicaciones .NET
    \item \textbf{JavaScript avanzado:} Manipulación DOM, eventos, temporizadores
    \item \textbf{Patrones de diseño:} Repository, Service, DTO, MVC
    \item \textbf{Testing y debugging:} Depuración de aplicaciones web complejas
\end{itemize}

\subsection*{Conclusiones}
El proyecto Quixo Web demuestra la capacidad para desarrollar una aplicación web completa utilizando tecnologías modernas. Se logró implementar todas las funcionalidades requeridas, con especial atención a la experiencia de usuario y robustez del sistema. La aplicación es escalable, mantenible y sigue las mejores prácticas de desarrollo web.

\subsection*{Posibles Mejoras Futuras}
\begin{itemize}
    \item \textbf{Autenticación de usuarios:} Sistema de login/registro
    \item \textbf{Juego en línea:} Multiplayer en tiempo real con SignalR
    \item \textbf{IA para oponente:} Implementar algoritmos para jugar contra computadora
    \item \textbf{App móvil:} Versión nativa usando .NET MAUI
    \item \textbf{Más estadísticas:} Gráficos avanzados, tendencias
\end{itemize}

\end{document}